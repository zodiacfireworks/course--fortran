%-----------------------------------------------------------------------------80
% CONTENT
%-----------------------------------------------------------------------------80
%Definición-------------------------------------------------------------------80

\subsection{Enteros, reales y complejos}

\begin{frame}[fragile]{Enteros}
\onslide<0->\textbf{Tipos de enteros}
 \begin{itemize}[<+(0)->]
  \item 
 \end{itemize}
\end{frame}

\begin{frame}[fragile]{Reales}
\onslide<0->\textbf{Tipos de reales}
 \begin{itemize}[<+(1)->]
  \item 
  \item La sintaxis para el tipo real es\\ 
   \centering real(kind=<np>)
  \vspace{6pt}
  \item []
   \begin{minted}[linenos,autogobble]{fortran}
    REAL(kind=p04) :: X04
    REAL(kind=p08) :: X08
    REAL(kind=p16) :: X16
    REAL(kind=p32) :: X32
   \end{minted}
  \item[] \rightline {\textit{Véase kind\_real.f95}}
 \end{itemize}
\end{frame}

