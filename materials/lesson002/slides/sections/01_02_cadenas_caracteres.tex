%-----------------------------------------------------------------------------80
% CONTENT
%-----------------------------------------------------------------------------80
%Definición-------------------------------------------------------------------80

\subsection{Cadenas de caracteres y valores lógicos}

\begin{frame}[fragile]{Cadenas de caracteres y valores lógicos}
\onslide<0->\textbf{Declaración de cadenas de caracteres}
\onslide<4->\textbf{Declaración de valores lógicos}
 \begin{itemize}[<+(0)->]
  \item La declaración de una variable de tipo character está dada por la sintaxis\\ 
   \centering CHARACTER[(len=<longitud>)],[<atributo(s)>][::]<variable(s)>[=<valor>]
  \vspace{6pt}
  \item []
   \begin{minted}[linenos,autogobble]{fortran}
    CHARACTER(kind=ascii, len=26) :: Alphabet
    CHARACTER(kind= ucs4, len=30) :: HelloWorld
   \end{minted}
  \item[] \rightline {\textit{Véase kind\_character.f95}}
  \item \item La declaración de una variable lógica está dada por\\ 
   \centering LOGICAL <variable(s)>
 \end{itemize}
\end{frame}

