%-----------------------------------------------------------------------------80
% CONTENT
%-----------------------------------------------------------------------------80
%Definición-------------------------------------------------------------------80

\subsection{Repeticiones: DO, DO WHILE, DO FROM X TO Y}

\begin{frame}[fragile]{Repeticiones: DO, DO WHILE, DO}  
 \begin{itemize}[<+(0)->]
  \item Permiten la ejecución repetitiva de una o más instrucciones bajo un número determinado de veces o una expresión lógica.
  \item Entre estas estructuras destacan
  \item [] \textbf{DO FROM X TO Y?} (desde-hasta) \\
   \begin{minted}[linenos,autogobble]{fortran}
    DO <variable> = <inicio>, <fin>, <incremento>
      bloque de declaraciones
    END DO 
   \end{minted}
  \item [] \textbf{DO WHILE} (mientras) \\ 
   \begin{minted}[linenos,autogobble]{fortran}
    DO WHILE <(expresiones logicas)> 
      bloque de declaraciones
    END DO 
   \end{minted}
  \item [] \textbf{DO?} (repetir-hasta)
   \begin{minted}[linenos,autogobble]{fortran}
    DO 
      bloque de declaraciones
       IF <expresion logica> EXIT
    END DO 
   \end{minted}
 \end{itemize}
\end{frame}



\begin{frame}[fragile]{Repeticiones: DO, DO WHILE, DO} 
 \begin{itemize}[<+(0)->]
  \item []
  \begin{minted}[linenos,autogobble]{fortran}
      ! Do finito
      ! Igual que un bucle FOR en otros lenguajes
            DO I = 3, N, 1
                Fibn = Fib2 + Fib1

                IF (Fibn < 0) THEN
                    WRITE(*, '(I5,2X,A)') I, "Overflow error!"
                    EXIT
                END IF

                WRITE(strFibn, '(I64)') Fibn
                WRITE(*, '(I5,2X,A)') I, ADJUSTL(strFibn)

                Fib1 = Fib2
                Fib2 = Fibn
            END DO
  \end{minted}
  \rightline {\textit{Véase fibonacci.f95}}
 \end{itemize}
\end{frame}



