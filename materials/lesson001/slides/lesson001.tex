%%----------------------------------------------------------------------------80
%% Preamble
%%----------------------------------------------------------------------------80
\documentclass[10pt,aspectratio=96]{beamer}
\usepackage[some]{background}
\usepackage[T1]{fontenc}
\usepackage{amsmath}
\usepackage{appendixnumberbeamer}
\usepackage{array}
\usepackage{booktabs}
\usepackage{colortbl}
\usepackage{fontawesome}
\usepackage{geometry}
\usepackage{graphicx}
\usepackage{hologo}
\usepackage{lipsum}
\usepackage{multirow}
\usepackage{pdflscape}
\usepackage{pgfcalendar}
\usepackage{pgfplots}
\usepackage{polyglossia}
\usepackage{tabularx}
\usepackage{xcolor-material}
\usepackage{xcolor}
\usepackage{xspace}

% \usepackage{float}
\usepackage{caption}
\usepackage[outputdir=build]{minted}
\usepackage{hyperref}


%%----------------------------------------------------------------------------80
%% Settings
%%----------------------------------------------------------------------------80
% Poligrossia pacakge settings -----------------------------------------------80
\setmainlanguage{spanish}


% Hyperref package settings --------------------------------------------------80
\hypersetup{
  pdftitle={Programaci\'{o}n en FORTRAN - Lecci\'{o}n 1},
  pdfauthor={Mart\'{i}n Josemar\'{i}a Vuelta Rojas},
  pdfpagelayout=OneColumn,
  pdfnewwindow=true,
  pdfdisplaydoctitle=true,
  pdfstartview=XYZ,
  plainpages=false,
  unicode=true,
  bookmarksnumbered=true,
  bookmarksopen=true,
  bookmarksopenlevel=3,
  breaklinks=true,
  colorlinks=true,
  pdfborder={0 0 0}
}


% Caption package settings ---------------------------------------------------80
\captionsetup[figure]{labelfont=bf,justification=centering}


% PGFPlots package settings --------------------------------------------------80
\pgfplotsset{compat=1.14}
\usepgfplotslibrary{dateplot}
\usepgfplotslibrary{groupplots}


% Beamer package settings ----------------------------------------------------80
\usetheme{metropolis}


% Minted package settings ----------------------------------------------------80
\usemintedstyle{manni}
\setminted{
  fontsize=\scriptsize,
  baselinestretch=1.15
}

%%----------------------------------------------------------------------------80
%% Customizations
%%----------------------------------------------------------------------------80

%----------FONDO DE CÓDIGO MINTED----------------%
\newminted[mintedbash]{bash}{linenos=true, autogobble=true, texcl=true, bgcolor=grey-300, fontsize=auto }

%---------------------------------------------%


%%----------------------------------------------------------------------------80
%% Custom commands definitions
%%----------------------------------------------------------------------------80

\makeatletter
\definecolor{green-600}{HTML}{43A047}
\definecolor{grey-300}{HTML}{E0E0E0}

%----------FONDO DE CÓDIGO MINTED----------------%
\renewenvironment{minted@colorbg}[1]
 {\def\minted@bgcol{#1}%
  \noindent
  \begin{lrbox}{\minted@bgbox}
  \begin{minipage}{\linewidth-2\fboxsep}}
 {\end{minipage}%
  \end{lrbox}%
  \setlength{\topsep}{\bigskipamount}% set the vertical space
  \trivlist\item\relax % ensure going to a new line
  \colorbox{\minted@bgcol}{\usebox{\minted@bgbox}}%
  \endtrivlist % close the trivlist
 }
 %-----------------------------------------------%

 \makeatother


%%----------------------------------------------------------------------------80
%% Document
%%----------------------------------------------------------------------------80
% Global document settings ---------------------------------------------------80
% Document body --------------------------------------------------------------80
\begin{document}
  % -*- BEGIN: Title Frame
  \title{Programación en FORTRAN}
  \subtitle{
    Nivel Básico - Sesión 1
  }
  \date{\today}
  \author{Martin Josemaría Vuelta Rojas}
  \institute{SoftButterfly}
  \maketitle
  % -*- END: Title Frame

  % -*- BEGIN: TOC
  \begin{frame}{Contenido}
    \setbeamertemplate{section in toc}[sections numbered]
    \tableofcontents[hideallsubsections]
  \end{frame}
  % -*- END: TOC

  %-----------------------------------------------------------------------------80
% SECTION TITLE
%-----------------------------------------------------------------------------80
\section{Preliminares}


%-----------------------------------------------------------------------------80
% CONTENT
%-----------------------------------------------------------------------------80
% Acerca del curso -----------------------------------------------------------80 

\subsection{Acerca del curso}
\begin{frame}[fragile]{Acerca del curso}
  \textbf{El curso básico de Fortran tiene por objetivo}
  \begin{itemize}[<+(1)->]
    \item Que obtengan un conocimiento sólido de las carácteristicas de Fortran.
    \item Que se familiarizen con el flujo de infromación en los programas desarrolados en Fortran.
    \item Que tengan una introducción nuevas caracteristicas de los estándares recientes Fortran 2003 y 2008.
    \item Que puedan contruir su entorno de darrollo para programar en Fortran cómodamente.
  \end{itemize}
\end{frame}


\begin{frame}[fragile]{Acerca del curso}
  \textbf{Metodología}
  \begin{itemize}[<+(1)->]
    \item Exposiciones dialogadas utilizando, apuntes, elementos de proyección fija.
    \item Talleres prácticos grupales en la elaboración de programas.
    \item Las clases serán eminentemente prácticas en una relación 70\% práctica - 30\% teórica.
  \end{itemize}
\end{frame}

\begin{frame}[fragile]{Acerca del curso}
  \textbf{Referencias}
  \begin{itemize}[<+(1)->]
    \item I. Chivers, J. Sleightholme, \textit{Introduction to Programming with Fortran. With Coverage of Fortran 90, 95, 2003, 2008 and 77}, Springer-Verlag London, 2012.
    \item Michael Metcalf, John Reid, Malcolm Cohen, \textit{Modern Fortran Explained}, Oxford University Press, USA 2011
    \item Morten Hjorth-Jensen, \textit{Computational Physics, Lecture Notes Fall 2015}, Department of Physics, University of Oslo. 2015.
  \end{itemize}
\end{frame}

\begin{frame}[fragile]{Acerca del curso}
  \textbf{Materiales}
  \begin{itemize}[<+(1)->]
    \item[] Repositorio del curso
    \begin{itemize}
      \item[\faGithub] \href{https://github.com/zodiacfireworks/course--fortran-basic}{https://github.com/zodiacfireworks/course--fortran-basic}
      \item[\faGithub] \href{https://github.com/zodiacfireworks/course--fortran-intermediate}{https://github.com/zodiacfireworks/course--fortran-intermediate}
    \end{itemize}
  \end{itemize}

  \onslide<5->\begin{alertblock}{¡Importante!}
    \begin{itemize}[<+(2)->]
      \item[] Cada alumno debera tener una cuenta en GitHub
      \begin{itemize}
        \item[\color{black}\faGithub] \href{https://github.com}{https://github.com}
      \end{itemize}
    \end{itemize}
  \end{alertblock}
\end{frame}


% Acerca del instructor ------------------------------------------------------80
\subsection{Acerca del instructor}
\begin{frame}[fragile]{Acerca del instructor}
  \begin{center}
    \textbf{Martín Josemaría Vuelta Rojas}
  \end{center}

  \begin{itemize}[<+(1)->]
    \item Software Developer
    \item Web Developer
    \item Investigador
    \item Programador
    \begin{itemize}
      \item En investigación:
        \begin{itemize}
          \item C, C++, Fortran, Python, R, Julia, Mathematica, Matlab, LaTeX
        \end{itemize}
      \item En web:
        \begin{itemize}
          \item HTML, CSS, JavaScript, Python
        \end{itemize}
      \item En mobile:
        \begin{itemize}
          \item Kotlin, Java, C++
        \end{itemize}
      \item Hobbie:
        \begin{itemize}
          \item Scala, Pixie, Clojure, Elixir, Haskel, Oz, Kotlin, ...
        \end{itemize}
    \end{itemize}
  \end{itemize}
\end{frame}

\begin{frame}[fragile]{Acerca del instructor}
  \begin{center}
    \textbf{Martín Josemaría Vuelta Rojas}
  \end{center}

  \begin{itemize}[<+(1)->]
    \item SoftButterfly
    \item HackSpace Perú
    \item Jupyter Notebook
    \item Fedora
    \item GNOME
    \item UNMSM
  \end{itemize}
\end{frame}

\begin{frame}[fragile]{Acerca del instructor}
  \begin{center}
    \textbf{Martín Josemaría Vuelta Rojas}
  \end{center}

  \begin{itemize}[<+(1)->]
    \item[\faMobile]   \begin{center}
      \href{tel:+51982042088}{+51 982 042 088}
    \end{center}
    \item[\faEnvelope] \begin{center}
      \href{mailto:martin.vuelta@gmail.com}{martin.vuelta@gmail.com}
    \end{center}
    \item[\faLinkedin] \begin{center}
      \href{https://www.linkedin.com/in/martinvuelta/}{martinvuelta}
    \end{center}
    \item[\faGithub]   \begin{center}
      \href{https://github.com/zodiacfireworks}{zodiacfireworks}
    \end{center}
  \end{itemize}
\end{frame}

  %-----------------------------------------------------------------------------80
% SECTION TITLE
%-----------------------------------------------------------------------------80
\section{Introducción histórica}


%-----------------------------------------------------------------------------80
% CONTENT
%-----------------------------------------------------------------------------80
% En el origen ---------------------------------------------------------------80
\subsection{En el origen}
\begin{frame}[fragile]{Orígenes}
  \textbf{En el origen ...}
  \begin{itemize}[<+(1)->]
      \item Código máquina en notación octal
      \item Conocimiento muy detallado del hardware
  \end{itemize}

  \onslide<4->\textbf{A inicios de los 50s}
  \begin{itemize}[<+(2)->]
    \item Assembler
    \item Menos laborioso que el código máquina
    \item Conocimiento detallado del hardware
  \end{itemize}

  \onslide<8->\textbf{El panorama general ...}
  \begin{itemize}[<+(3)->]
    \item Conocimiento del harware
    \item Facilidad de cometer errores
    \item Encontrar los errores en los programas era bastante difícil
  \end{itemize}
\end{frame}


\begin{frame}[fragile]{Orígenes}
  \textbf{Génesis 1953 ...}
  \begin{itemize}[<+(1)->]
    \item[] John  Backus envía una carta a su jefe en IBM pidiendo permiso para
      investigar una \textit{mejor forma} de programar las computadoras. La
      carta contenia un esboo de projecto con un tiempo de desarrollo de 6
      meses.
    \item[] Así emepzo el proyecto que daría origen a Fortran
  \end{itemize}
\end{frame}


\begin{frame}[fragile]{Orígenes}
  \begin{quote}
    ``The project completion was always six months away!''\\
     John Backus
  \end{quote}
\end{frame}


\begin{frame}[fragile]{Orígenes}
  \textbf{1957}
  \begin{itemize}[<+(1)->]
    \item[] En febrero FORTRAN, el primer lenguaje de programación de alto nivel, fue anunciado al mundo por John Backus y su equipo de IBM en la Western Joint Computer Conference celebrada en Los Ángeles.
    \item[] A mediados de abril de 1957 tuvo lugar la primera entrega del compilador de FORTRAN para IBM 704 a Westinghouse Bettis para su uso en el diseño de reactores nucleares.
  \end{itemize}
\end{frame}


\begin{frame}[fragile]{Orígenes}
  \textbf{Las versiones de FORTRAN}
  \begin{description}[<+(1)->]
    \item[1957] FORTRAN I
    \item[1958] FORTRAN II
    \item[1958] FORTRAN III (No disponible al público)
    \item[1961] FORTRAN IV (Una versión mejorada de FORTRAN II)
      \end{description}
\end{frame}


\begin{frame}[fragile]{Orígenes}
    \textbf{Algoritmo TPK en FORTRAN 0}
    \begin{itemize}
    \item []
     \begin{minted}[linenos,autogobble]{fortran}
             DIMENSION A(11)
             READ A
      2      DO 3,8,11 J=1,11
      3      I=11-J
             Y=SQRT(ABS(A(I+1)))+5*A(I+1)**3
             IF (400>=Y) 8,4
      4      PRINT I,999.
             GOTO 2
      8      PRINT I,Y
      11     STOP
     \end{minted}
    \end{itemize}
\end{frame}


\begin{frame}[fragile]{Orígenes}
    \textbf{Algoritmo TPK en FORTRAN I}
    \begin{itemize}
    \item []  
     \begin{minted}[linenos,autogobble]{fortran}
      C      THE TPK ALGORITHM
      C      FORTRAN I STYLE
             FUNF(T)=SQRTF(ABSF(T))+5.0*T**3
             DIMENSION A(11)
        1    FORMAT(6F12.4)
             READ 1,A
             DO 10 J=1,11
             I=11-J
             Y=FUNF(A(I+1))
             IF(400.0-Y)4,8,8
        4    PRINT 5,I
        5    FORMAT(I10,10H TOO LARGE)
             GOTO 10
        8    PRINT 9,I,Y
        9    FORMAT(I10,F12.7)
       10    CONTINUE
             STOP 52525
     \end{minted}
    \end{itemize}
\end{frame}


% La estandarización ---------------------------------------------------------80
\subsection{La estandarización}
\begin{frame}[fragile]{La estándarización}
  \begin{itemize}[<+(1)->]
    \item[1962] El primer comité de estandarización de la ASA (Ahora ANSI) se re reune.
    \item[1966] Publicación del ANSI X3.91966 (FORTRAN 66), el primer estándar.
    \item[1978] Publicación del ANSI X3.91978 (FORTRAN 77), tambien publicado como ISO 1539:1980.
    \item[1991] ISO/IEC 1539:1991 (Fortran 90)
    \item[1997] ISO/IEC 1539­1:1997 (Fortran 95)
    \item[2004] ISO/IEC 1539­1:2004 (Fortran 2003)
    \item[2010] ISO/IEC 1539­1:2010 (Fortran 2008)
  \end{itemize}
\end{frame}


\begin{frame}[fragile]{La estándarización}
    \textbf{Algoritmo TPK en FORTRAN 77}
    \begin{itemize}
    \item []   
     \begin{minted}[linenos,autogobble]{fortran}
             PROGRAM TPK
     C       THE TPK ALGORITHM
     C       FORTRAN 77 STYLE
             REAL A(0:10)
             READ (5,*) A
             DO 10 I = 10, 0, -1
                     Y = FUN(A(I))
                     IF ( Y . LT. 400) THEN
                              WRITE(6,9) I,Y
       9                      FORMAT(I10. F12.6)
                     ELSE
                              WRITE (6,5) I
       5                      FORMAT(I10,' TOO LARGE')
                     ENDIF
      10    CONTINUE
            END

            REAL FUNCTION FUN(T)
            REAL T
            FUN = SQRT(ABS(T)) + 5.0*T**3
            END
     \end{minted}
    \end{itemize}
\end{frame}


\begin{frame}[fragile]{La estándarización}
    \textbf{Algoritmo TPK en FORTRAN 90}
    \begin{itemize}
    \item []  
     \begin{minted}[linenos,autogobble]{fortran}
            PROGRAM TPK
     !      The TPK Algorithm
     !      Fortran 90 style
            IMPLICIT NONE
            INTEGER          :: I
            REAL                       :: Y
            REAL, DIMENSION(0:10)      :: A
            READ (*,*) A
            DO I = 10, 0, -1           ! Backwards
                    Y = FUN(A(I))
                    IF ( Y < 400.0 ) THEN
                           WRITE(*,*) I, Y
                    ELSE
                           WRITE(*,*) I, ' Too large'
                    END IF
            END DO
            CONTAINS                   ! Local function
                    FUNCTION FUN(T)
                    REAL  :: FUN
                    REAL, INTENT(IN) :: T
                    FUN = SQRT(ABS(T)) + 5.0*T**3
                    END FUNCTION FUN
            END PROGRAM TPK
     \end{minted}
    \end{itemize}
\end{frame}

\begin{frame}[fragile]{La estándarización}
    \textbf{Algoritmo TPK en FORTRAN 90}
    \begin{itemize}
    \item []  
     \begin{minted}[linenos,autogobble]{fortran}
            module Functions  
            public :: fun
            contains
               function fun(t) result (r)
                  real, intent(in) :: t
                  real  :: r
                  r = sqrt(abs(t)) + 5.0*t**3
               end function fun
            end module Functions

            program TPK
      !     The TPK Algorithm
      !     F95 style
            use Functions
            integer       :: i
            real                  :: y
            real, dimension(0:10) :: a
            read *, a
            do i = 10, 0, -1      ! Backwards
               y = fun(a(i))
               if ( y < 400.0 ) then
                  print *, i, y
               else
                  print *, i, " Too large"
               end if
            end do
            end program TPKTPK
     \end{minted}
    \end{itemize}
\end{frame}


% Actualidad -----------------------------------------------------------------80
\subsection{En la actualidad}
\begin{frame}[fragile]{En la actualidad ...}
  \begin{quote}
    ``I don't know what the programming language of the year 2000 will look like, but I know it will be called FORTRAN.''\\
    Charles Anthony Richard Hoare\\
    Circa 1982
  \end{quote}
\end{frame}


\begin{frame}[fragile]{En la actualidad ...}
  \textbf{Aplicaciones}
  \begin{itemize}[<+(1)->]
    \item Predicción del clima
    \item Análisis de datos de sísmicos para la exploración de depositos de gas y petroleo
    \item Análisis fianciero
    \item Simulacion de choques vehiculares
    \item Análisis de datos de sondas espaciales
    \item Modelación de armas nucleares
    \item Dinámica de fluidos computacionales
    \item ``Numerical Wind Tunnel''
  \end{itemize}
\end{frame}


\begin{frame}[fragile]{En la actualidad ...}
  \begin{itemize}[<+(1)->]
    \item Intel
    \item IBM
    \item NVIDIA
  \end{itemize}
\end{frame}


% Futuro ---------------------------------------------------------------------80
\subsection{El futuro}
\begin{frame}[fragile]{El futuro}
  \begin{itemize}[<+(1)->]
    \item \textbf{WG5: }\url{https://wg5-fortran.org/}
    \item \textbf{Fortran Wiki: }\url{http://fortranwiki.org/}
    \item \textbf{WG5: }\url{https://wg5-fortran.org/}
  \end{itemize}
\end{frame}


  %-----------------------------------------------------------------------------80
% SECTION TITLE
%-----------------------------------------------------------------------------80
\section{Entorno de desarrollo}


%-----------------------------------------------------------------------------80
% CONTENT
%-----------------------------------------------------------------------------80
% Definición -----------------------------------------------------------------80
\subsection{Introducción}
\begin{frame}[fragile]{Entorno de desarrollo}
  \begin{figure}
      \includegraphics[width=1\textwidth]{./resources/IDE-VSCODE.png}
      \caption{Entorno de desarrollo VS Code}
  \end{figure}
\end{frame}

\begin{frame}[fragile]{Entorno de desarrollo}
  \textbf{Que tiene un entorno de desarrollo}
  \begin{itemize}[<+(1)->]
    \item Editor de código
    \item Compiladores o intérpretes
    \item Debugger
    \item Otras utilidades
  \end{itemize}
\end{frame}


\begin{frame}[fragile]{Entorno de desarrollo}
  \textbf{Editor de código}
  \begin{itemize}[<+(1)->]
    \item VS Code + Fortran Package [{\color{green-600}\faCheck}]
    \item Sublime Text
    \item Atom
    \item Vim
    \item Emacs
    \item ...
  \end{itemize}
\end{frame}


\begin{frame}[fragile]{Entorno de desarrollo}
  \textbf{Compilador}
  \begin{itemize}[<+(1)->]
    \item GFortran [{\color{green-600}\faCheck}]
    \item Intel
    \item IBM
    \item Oracle
    \item ...
  \end{itemize}
\end{frame}


\begin{frame}[fragile]{Entorno de desarrollo}
  \textbf{Debuggers}
  \begin{itemize}[<+(1)->]
    \item gdb [{\color{green-600}\faCheck}]
    \item idb
    \item ddd
    \item totalview
    \item ...
  \end{itemize}
\end{frame}

  %-----------------------------------------------------------------------------80
% CONTENT
%-----------------------------------------------------------------------------80
\subsection{Entorno de desarrollo en Linux}


\begin{frame}[fragile]{Instalación de editor de código}
  \textbf{VS Code en Ubuntu y derivados}
  \begin{itemize}[<+(1)->]
   \item Descargar el paquete .deb desde \url{https://code.visualstudio.com/download}
   \item Instalar el paquete desde la terminal
    \begin{mintedbash}
      sudo dpkg -i <nombre del archivo>.deb
      sudo apt-get install -f 
    \end{mintedbash} 
   \item Actualizar el paquete e instalar code
     \begin{mintedbash}
       sudo apt-get update
       sudo apt-get install code # o también  "code-insiders"
     \end{mintedbash}
  \end{itemize}
\end{frame}

\begin{frame}[fragile]{Instalación de editor de código}
  \textbf{VS Code en Fedora/CentOS}
  \begin{itemize}[<+(1)->]
   \item Descargar el paquete .rpm desde \url{https://code.visualstudio.com/download}
   \item Instalar el paquete desde la terminal según la versión Fedora/CentOs (yum o dnf)
    \begin{mintedbash}
      yum check-update
      sudo yum install <nombre del archivo>.rpm 
    \end{mintedbash}
    \begin{mintedbash}
      dnf check-update
      sudo dnf install <nombre del archivo>.rpm 
    \end{mintedbash}
  \end{itemize}
\end{frame}

\begin{frame}[fragile]{Fortran en VS Code}
  \begin{itemize}
   \item  En la pestaña 'Extensiones' (Ctrl+Shift+X), instalar 'Modern Fortran 0.6.2' encontrado con el buscador y recargar la extensión. 
  \end{itemize}
   \begin{figure}
    \includegraphics[width=0.75\textwidth]{./resources/Moder_fortran.png}
    \caption*{Extensión Modern Fortran 0.6.2 en VS Code}
   \end{figure}
\end{frame}

\begin{frame}[fragile]{Instalación de compilador y debuger}
  \textbf{Compilador y debugger en Ubuntu y derivados}
\begin{itemize}[<+(1)->] 
\item \textbf{Instalación de Gfortran}
    \begin{mintedbash} 
       sudo apt-get install gfortran
    \end{mintedbash}
  \item	\textbf{Instalación del paquete binutils}
    \begin{mintedbash}
       sudo apt-get update
       sudo apt-get install binutils
    \end{mintedbash}
\item	\textbf{Instalación del paquete build-essential}
    \begin{mintedbash}
       sudo apt-get update
       sudo apt-get install build-essential
  	\end{mintedbash}
\end{itemize}
\end{frame}


\begin{frame}[fragile]{Instalación de compilador y debuger}
  \textbf{Compilador y debugger en Fedora/CentOS}
\begin{itemize}[<+(1)->] 
\item	\textbf{Instalación de Gfortran}
    \begin{mintedbash}
       yum install gcc-gfortran
   	\end{mintedbash}
  \item	\textbf{Instalación del paquete Development tools}
    \begin{mintedbash}
       yum clean all
       yum groupinstall "Development tools" 
	  \end{mintedbash}
\end{itemize}
\end{frame}


  %-----------------------------------------------------------------------------80
% CONTENT
%-----------------------------------------------------------------------------80
\subsection{Entorno de desarrollo Windows}


\begin{frame}[fragile]{Instalación de editor de código}
  \textbf{VS Code en Windows}
  \begin{itemize}[<+(1)->]
    \item Descargar el instalador desde \url{https://code.visualstudio.com/download}
  \end{itemize}  
    \begin{figure}
      \includegraphics[width=1\textwidth]{./resources/1.png}
    \end{figure}

\end{frame}


\begin{frame}[fragile]{Instalación de editor de código}
  \textbf{VS Code en Windows}
  \begin{itemize}[<+(1)->]
    \item Iniciado el instalador, click en siguiente.
  \end{itemize} 
    \begin{figure}
      \includegraphics[width=1\textwidth]{./resources/2.png}
    \end{figure}
\end{frame}

\begin{frame}[fragile]{Instalación de editor de código}
  \textbf{VS Code en Windows}
  \begin{itemize}[<+(1)->]
    \item En la siguiente ventana, marcar de preferencia las opciones mostradas.
  \end{itemize} 
  \begin{figure}
    \includegraphics[width=1\textwidth]{./resources/3.png}
  \end{figure}
\end{frame}

\begin{frame}[fragile]{Instalación de editor de código}
  \textbf{VS Code en Windows}
  \begin{itemize}[<+(1)->]
    \item Click en siguiente hasta finalizar la instalación. Se ejecutará el programa.
  \end{itemize}
  \begin{figure}
    \includegraphics[width=1\textwidth]{./resources/4.png}
  \end{figure}
\end{frame}

\begin{frame}[fragile]{Instalación de compilador y debuger}
 \textbf{Instalación del compilador TDM-GCC}
 \begin{itemize}[<+(1)->]
  \item Descargar el paquete TDM-GCC según la versión de windows (32-64 bits) en \url{http://tdm-gcc.tdragon.net/}
  \item asdasd
 \end{itemize}
\end{frame}


  %-----------------------------------------------------------------------------80
% SECTION TITLE
%-----------------------------------------------------------------------------80
\section{Conceptos básicos}


%-----------------------------------------------------------------------------80
% CONTENT
%-----------------------------------------------------------------------------80
%Definición-------------------------------------------------------------------80


\subsection{Hello world!}
\begin{frame}[fragile]{Hello world! y el proceso de compilación}
 \textbf{Hello world!}
  \begin{itemize}[<+(1)->]
   \item Se llama programa a un conjunto de instrucciones, realizadas computacionalmente en un tiempo determinado, aplicadas en la introducción, procesamiento o salida de datos.   
   \item Un programa en fortran tiene la siguiente forma:
   \vspace{6pt}
   \item []
    \begin{minted}[linenos,autogobble]{fortran}
     PROGRAM hello_world
        WRITE(*, *) Message
     END PROGRAM hello_world
    \end{minted}
  \end{itemize}
\end{frame}

\begin{frame}[fragile]{Hello world! y el proceso de compilación}
 \textbf{Scientific Hello world!}
  \begin{itemize}[<+(1)->]
   \vspace{6pt}
   \item []
    \begin{minted}[linenos,autogobble]{fortran}
    PROGRAM hello_world
    IMPLICIT NONE

    ! Angulo de entrada
    REAL(KIND=4) :: theta
    
    ! Resultado de aplicar la función seno 
    REAL(KIND=4) :: sin_of_theta
    
    ! Mensaje
    CHARACTER(len=*), PARAMETER :: Message = 'Hello World'

    WRITE(*, *) 'Ingrese un ángulo [rad]: '
    READ(*, *) theta
    sin_of_theta = SIN(theta)
    
    WRITE(*, *) Message
    WRITE(*, *) "sin(", theta, ") = ", sin_of_theta
    END PROGRAM hello_world
    \end{minted}
  \end{itemize}
\end{frame}


  %-----------------------------------------------------------------------------80
% CONTENT
%-----------------------------------------------------------------------------80

\subsection{Proceso de compilación}

\begin{frame}[fragile]{Proceso de compilación}
  \begin{itemize}[<+(1)->]
  \item  El proceso de compilación se puede resumir en dos pasos
  \hspace{1cm} \item [-] Compilación
  \hspace{1cm} \item [-] Enlazado

  \item El proceso de compilación en fortran posee la siguiente sintaxis:
   \begin{mintedbash}
    fcomp [options] file1 [file2] [...] [fileN]
   \end{mintedbash}
  \item [-] fcomp $\Rightarrow$ denota el comando para llamar al compilador. (gfortran, ifort, ...)
  \item [-] options $\Rightarrow$ opciones que permite el compilador. (-o, -f, -c, ...)
  \item [-] file $\Rightarrow$ denota el archivo con su respectiva extensión (.f90, .f95, .o, ...)
  \item Finalmente se obtiene un producto final que comúnmente es un archivo ejeutable.
\end{itemize}
\end{frame}


\begin{frame}[fragile]{Compilación}
\textbf{Compilador}
  \begin{itemize}[<+(1)->]
  \item Un compialdor es un programa que transforma el codigo escrito
  en un lenguaje de programación (lenguaje fuente) en otro archivo escrito
  en otro lenguaje de programacion (lenguaje objetivo.

  \item El lenguaje fuente es con frecuencia un lenguaje legible para los
  humanos, facil de leer e interpretar, como FORTRAN.

  \item Por otra parte el lenguaje objetivo es el lengaje máquina, el que
  se emplea para darle instrucciones que la maquina pueda entender y
  ejeutar.
 \end{itemize}
\end{frame}


\begin{frame}[fragile]{Compilación}

  \begin{figure}
    \includegraphics[width=1\textwidth]{./resources/compilation_op.png}
    \caption{Diagrama de bloques del proceso de compilación}
   \end{figure}
\end{frame}

  %% %-----------------------------------------------------------------------------80
% CONTENT
%-----------------------------------------------------------------------------80

  %% %-----------------------------------------------------------------------------80
% CONTENT
%-----------------------------------------------------------------------------80

\subsection{Proceso de compilación}

\begin{frame}[fragile]{Productos finales}
\textbf{Ejecutables? :v}
  \begin{itemize}[<+(1)->]
  \item Programa  escrito en un lenguaje de programación, que a su vez, traduce y genera otro programa
en otro lenguaje de programación (código máquina), ambos equivalentes.
  \item En principio, al usar un compilador, se busca traducir y simplificar un lenguaje de mayor complejidad a uno mucho más cotidiano y manejable en términos informáticos.
 \end{itemize}
\end{frame}
  %-----------------------------------------------------------------------------80
% SECTION TITLE|
%-----------------------------------------------------------------------------80

\section{Formatos de escritura en Fortran}

%-----------------------------------------------------------------------------80
% CONTENT
%-----------------------------------------------------------------------------80
%Definición-------------------------------------------------------------------80

\begin{frame}[fragile]{Formatos de escritura de un programa de Fortran}
  \begin{itemize}[<+(1)->]
  \item Se crea el código fuente en un editor de textos y se guarda con su respectiva extensión (.f, .f90, .f95)
  \item [] \begin{mintedbash} 
            editor nombre.extensión
           \end{mintedbash}
  \item Se emplean caracteres ASCI en el código fuente (alfanuméricos, símbolos y espacio)
  \item No hay distinción entre mayúsculas y minúsculas.
  \item Los caracteres propios del español no pueden ser utilizadas en las instrucciones (á, é, ñ)
  \end{itemize}
\end{frame}

\begin{frame}[fragile]{Formatos de escritura de un programa de Fortran}
 \textbf{Formato libre}
  \begin{itemize}[<+(1)->]
   \item Aplicable a versiones Fortran90 en adelante de extensión f90, f95...  
   \item Permite un máximo de 132 caracteres.
   \item Se emplea el caracter (!) para empezar un comentario, sin que este se tome en cuentra durante la compilación.
   \item [] 
    \begin{minted}[linenos,autogobble]{fortran}
    <instruccion> ! <comentario>
    \end{minted}
   \item Se emplea el caracter (;) para separar dos instrucciones.
   \item [] 
    \begin{minted}[linenos,autogobble]{fortran}
    <instruccion>; <instruccion>
    \end{minted}
  \end{itemize}
\end{frame} 

\begin{frame}[fragile]{Formatos de escritura de un programa de Fortran}
 \begin{itemize}[<+(0)->]
 \item En caso falte espacio en una línea, el caracter (\&) permite continuar las instrucciones en la siguiente línea, colocándolo al final de la línea anterior y al inicio de la línea por comenzar.
 \item [] 
  \begin{minted}[linenos,autogobble]{fortran}
        <instruccion 1> &
      & <instruccion 1> &
      & <instruccion 1>
        <instruccion 2>
  \end{minted}
 \item Las variables y unidades programaticas deben de ser nobradas empleando una combinacion de maximo 31 caracteres siendo el primer caracter una letra. 
 \item []
  \begin{minted}[linenos,autogobble]{fortran} 
      PROGRAM TPK95
  \end{minted} 
 \end{itemize}
\end{frame}


\begin{frame}[fragile]{Formatos de escritura de un programa de Fortran}
 \textbf{Formato fijo}
  \begin{itemize}[<+(1)->]
   \item Aplicable a versiones anteriores a Fortran90 de extensión f, for...
   \item Permite un máximo de 72 caracteres.
   \item Se emplea el caracter (C) para comenzar un comentario.
   \item []
    \begin{minted}[linenos,autogobble]{fortran} 
c --- Comentario
    \end{minted} 
   \item Se permite una sola instrucción por línea, comenzando a partir del séptimo caracter.
   \item []
    \begin{minted}[linenos,autogobble]{fortran} 
c23456789
      <Instruccion>
    \end{minted}
  \end{itemize}
\end{frame}

\begin{frame}[fragile]{Formatos de escritura de un programa de Fortran}
   \begin{itemize}[<+(0)->]
   \item La continuación de una instrucción en una siguiente línea está dada por algún caracter no alfanumérico en el sexto caracter.
    \begin{minted}[linenos,autogobble]{fortran} 
c23456789
      <Instruccion muy larga>
     &<Continuacion de instruccion muy larga>
    \end{minted}
   \item Los primeros cinco caracteres en una línea de instrucción se pueden emplear en caso se requiera una etiqueta (número positivo diferente de cero con un máximo de cinco dígitos).
    \begin{minted}[linenos,autogobble]{fortran} 
c23456789
75289 <Instruccion muy larga>
     &<Continuacion de instruccion muy larga>
    \end{minted}
   \item Las variables deben de ser nobradas empleando una combinacion de maximo 6 caracteres siendo el primer caracter una letra.
    \begin{minted}[linenos,autogobble]{fortran} 
c23456789
   12 REAL*8 FUNCTION Fib
    \end{minted}
  \end{itemize}
\end{frame}
  %-----------------------------------------------------------------------------80
% SECTION TITLE|
%-----------------------------------------------------------------------------80

\section{Estructura de un programa en Fortran}

%-----------------------------------------------------------------------------80
% CONTENT
%-----------------------------------------------------------------------------80
%Definición-------------------------------------------------------------------80

\subsection{Programa en Fortran}
\begin{frame}[fragile]{Estructura de un programa en Fortran}
  \begin{itemize}[<+(1)->]
   \item Un programa en Fortran está ordenado lógica y jerarquicamente en unidades programáticas. 
   \item Tiene una unidad principal, llamada programa principal, la cual contiene las instrucciones que definen el objetivo del programa.
   \item Es posible recurrir a subprogramas, que agrupados constituyen una instrucción del programa principal. 
   \item [] 
   
   \begin{minted}[linenos,autogobble]{fortran}
PROGRAM         ejemplo   !nombre del programa          !sentencia no ejecutable
IMPLICIT NONE             !declara todas las variables  !sentencia no ejecutable                               
INTEGER         i         !declaración variables        !sentencia no ejecutable
                i = 3     !código del programa          !sentencia ejecutable
WRITE(*,*)      ’ i =’,i  !código del programa          !sentencia ejecutable
END PROGRAM ejemplo                                     !sentencia ejecutable
  \end{minted}
\end{itemize} 
\end{frame}

\begin{frame}[fragile]{Estructura de un programa en Fortran}
 \begin{figure}
  \includegraphics[width=1\textwidth]{./resources/structurep.png}
  \caption{Esquema de la estructura de un programa en Fotran}
 \end{figure}
\end{frame}
  %-----------------------------------------------------------------------------80
% SECTION TITLE|
%-----------------------------------------------------------------------------80

\section{Expresividad: el algoritmo TPK}

%-----------------------------------------------------------------------------80
% CONTENT
%-----------------------------------------------------------------------------80
%Definición-------------------------------------------------------------------80

\subsection{EL algoritmo TPK}

\begin{frame}[fragile]{El algoritmo TPK}
 \begin{itemize}[<+(1)->]
 \item El algoritmo TPK es el "ejemplo inicial" para una secuencia de aprendizaje en programación. Fue empleado en el informe técnico \href{http://bitsavers.org/pdf/stanford/cs_techReports/STAN-CS-76-562_EarlyDevelPgmgLang_Aug76.pdf}{The Early Development of Programming Languages}  por Donald Knuth y Luis Trabb del departamento de ciencia computacional de la Universidad de Stanford.   
 \end{itemize}
\end{frame}

\begin{frame}[fragile]{El algoritmo TPK}
 \begin{itemize}[<+(0)->]
    \item []
    \begin{minted}[linenos,autogobble]{fortran}  
    MODULE Functions
    PUBLIC :: fun
    CONTAINS
        FUNCTION fun(t) result (r)
            REAL, INTENT(IN) :: t
            REAL  :: r
            r = SQRT(ABS(t)) + 5.0*t**3
        END FUNCTION fun
    END MODULE Functions

    PROGRAM TPK95
    ! The TPK Algorithm
    ! F style
    USE Functions
    INTEGER :: i
    REAL    :: y
    REAL, DIMENSION(0:10) :: a
    READ *, a
    DO i = 10, 0, -1 ! Bucle DO con contador hacia atrás
        y = fun(a(i))
        IF ( y < 400.0 ) THEN
            PRINT *, i, y
        ELSE
            PRINT *, i, " Too large"
        END IF
    END DO
    END PROGRAM tpk95
  \end{minted}
   \end{itemize} 
\end{frame}
\end{document}
