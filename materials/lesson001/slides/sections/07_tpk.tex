%-----------------------------------------------------------------------------80
% SECTION TITLE|
%-----------------------------------------------------------------------------80

\section{Expresividad: el algoritmo TPK}

%-----------------------------------------------------------------------------80
% CONTENT
%-----------------------------------------------------------------------------80
%Definición-------------------------------------------------------------------80

\subsection{EL algoritmo TPK}

\begin{frame}[fragile]{El algoritmo TPK}
 \begin{itemize}[<+(1)->]
 \item El algoritmo TPK es el "ejemplo inicial" para una secuencia de aprendizaje en programación. Fue empleado en el informe técnico \href{http://bitsavers.org/pdf/stanford/cs_techReports/STAN-CS-76-562_EarlyDevelPgmgLang_Aug76.pdf}{The Early Development of Programming Languages}  por Donald Knuth y Luis Trabb del departamento de ciencia computacional de la Universidad de Stanford.   
 \end{itemize}
\end{frame}

\begin{frame}[fragile]{El algoritmo TPK}
 \begin{itemize}[<+(0)->]
    \item []
    \begin{minted}[linenos,autogobble]{fortran}  
    MODULE Functions
    PUBLIC :: fun
    CONTAINS
        FUNCTION fun(t) result (r)
            REAL, INTENT(IN) :: t
            REAL  :: r
            r = SQRT(ABS(t)) + 5.0*t**3
        END FUNCTION fun
    END MODULE Functions

    PROGRAM TPK95
    ! The TPK Algorithm
    ! F style
    USE Functions
    INTEGER :: i
    REAL    :: y
    REAL, DIMENSION(0:10) :: a
    READ *, a
    DO i = 10, 0, -1 ! Bucle DO con contador hacia atrás
        y = fun(a(i))
        IF ( y < 400.0 ) THEN
            PRINT *, i, y
        ELSE
            PRINT *, i, " Too large"
        END IF
    END DO
    END PROGRAM tpk95
  \end{minted}
   \end{itemize} 
\end{frame}