%-----------------------------------------------------------------------------80
% CONTENT
%-----------------------------------------------------------------------------80

\subsection{Formatos de escritura de un programa de Fortran}

\begin{frame}[fragile]{Formatos de escritura de un programa de Fortran}
  \begin{itemize}[<+(1)->]
  \item Se crea el código fuente en un editor de textos y se guarda con su respectiva extensión (.f, .f90, .f95)
  \item [] \begin{mintedbash} 
            editor nombre.extensión
           \end{mintedbash}
  \item Se emplean caracteres ASCI en el código fuente (alfanuméricos, símbolos y espacio)
  \item No hay distinción entre mayúsculas y minúsculas.
  \item Los caracteres propios del español no pueden ser utilizadas en las instrucciones (á, é, ñ)
  \end{itemize}
\end{frame}

\begin{frame}[fragile]{Formatos de escritura de un programa de Fortran}
 \textbf{Formato libre}
  \begin{itemize}[<+(1)->]
   \item Aplicable a versiones de Fortran 90 en adelante, de extensión f.90, f.95, ...
   \item Permite un máximo de 132 caracteres.
   \item Se emplea el caracter (!) para empezar un comentario, sin que este se tome en cuentra durante la compilación.
   \item Se emplea el caracter (;) para separar dos instrucciones.
   \item En caso falte espacio en una línea, el caracter (\&) permite continuar las instrucciones en la siguiente línea, colocándolo al final de la línea anterior y al inicio de la línea por comenzar.
   \item Aquí falta el apartado 5 !!!!!!!!!
   \item 
   \item 
  \end{itemize}
\end{frame}

\begin{frame}[fragile]{Formatos de escritura de un programa de Fortran}
 \textbf{Formato fijo}
  \begin{itemize}[<+(1)->]
   \item Aplicable a versiones anteriores a Fortran 90, de extensión .f, .for, ...
   \item Permite un máximo de 72 caracteres.
   \item Se emplea el caracter (C) para comenzar un comentario. 
   \item Se permite una sola instrucción por línea, comenzando a partir del séptimo caracter.
   \item La continuación de una instrucción en una siguiente línea está dada por algún caracter no alfanumérico en el sexto caracter. 
   \item Los primeros cinco caracteres en una línea de instrucción se pueden emplear en caso se requiera una etiqueta (número positivo diferente de cero con un máximo de cinco dígitos).
   \item Apartado 6!!!
  \end{itemize}
\end{frame}

