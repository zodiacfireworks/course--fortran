%-----------------------------------------------------------------------------80
% CONTENT
%-----------------------------------------------------------------------------80

\subsection{Estructura de un programa en Fortran}

\begin{frame}[fragile]{Estructura de un programa en Fortran}
  \begin{itemize}[<+(1)->]
    \item Un programa en Fortran está ordenado lógica y jerarquicamente en unidades programáticas. 
    \item Tiene una unidad principal, llamada programa principal, la cual contiene las instrucciones que definen el objetivo del programa.
    \item Es posible recurrir a subprogramas, que agrupados constituyen una instrucción del programa principal. 
    \item [] 
      \begin{figure}
      \includegraphics[width=1\textwidth]{./resources/structurep.png}
      \caption{Esquema estructura de un programa en Fotran}
      \end{figure}
 
 \end{itemize} 
  \begin{minted}[linenos,autogobble]{fortran}
PROGRAM         ejemplo   !nombre del programa          !sentencia no ejecutable
IMPLICIT NONE             !declara todas las variables  !sentencia no ejecutable                               
INTEGER         i         !declaración variables        !sentencia no ejecutable
                i = 3     !código del programa          !sentencia ejecutable
WRITE(*,*)      ’ i =’,i  !código del programa          !sentencia ejecutable
END PROGRAM                                             !sentencia ejecutable
  \end{minted}

\end{frame}