%-----------------------------------------------------------------------------80
% SECTION TITLE
%-----------------------------------------------------------------------------80
\section{Introducción histórica}


%-----------------------------------------------------------------------------80
% CONTENT
%-----------------------------------------------------------------------------80
% En el origen ---------------------------------------------------------------80
\subsection{En el origen}
\begin{frame}[fragile]{Orígenes}
  \textbf{En el origen ...}
  \begin{itemize}[<+(1)->]
      \item Código máquina en notación octal
      \item Conocimiento muy detallado del hardware
  \end{itemize}

  \onslide<4->\textbf{A inicios de los 50s}
  \begin{itemize}[<+(2)->]
    \item Assembler
    \item Menos laborioso que el código máquina
    \item Conocimiento detallado del hardware
  \end{itemize}

  \onslide<8->\textbf{El panorama general ...}
  \begin{itemize}[<+(3)->]
    \item Conocimiento del harware
    \item Facilidad de cometer errores
    \item Encontrar los errores en los programas era bastante difícil
  \end{itemize}
\end{frame}


\begin{frame}[fragile]{Orígenes}
  \textbf{Génesis 1953 ...}
  \begin{itemize}[<+(1)->]
    \item[] John  Backus envía una carta a su jefe en IBM pidiendo permiso para
      investigar una \textit{mejor forma} de programar las computadoras. La
      carta contenia un esboo de projecto con un tiempo de desarrollo de 6
      meses.
    \item[] Así emepzo el proyecto que daría origen a Fortran
  \end{itemize}
\end{frame}


\begin{frame}[fragile]{Orígenes}
  \begin{quote}
    ``The project completion was always six months away!''\\
     John Backus
  \end{quote}
\end{frame}


\begin{frame}[fragile]{Orígenes}
  \textbf{1957}
  \begin{itemize}[<+(1)->]
    \item[] En febrero FORTRAN, el primer lenguaje de programación de alto nivel, fue anunciado al mundo por John Backus y su equipo de IBM en la Western Joint Computer Conference celebrada en Los Ángeles.
    \item[] A mediados de abril de 1957 tuvo lugar la primera entrega del compilador de FORTRAN para IBM 704 a Westinghouse Bettis para su uso en el diseño de reactores nucleares.
  \end{itemize}
\end{frame}


\begin{frame}[fragile]{Orígenes}
  \textbf{Las versiones de FORTRAN}
  \begin{description}[<+(1)->]
    \item[1957] FORTRAN I
    \item[1958] FORTRAN II
    \item[1958] FORTRAN III (No disponible al público)
    \item[1961] FORTRAN IV (Una versión mejorada de FORTRAN II)
      \end{description}
\end{frame}


\begin{frame}[fragile]{Orígenes}
    \textbf{Algoritmo TPK en FORTRAN 0}
    \begin{itemize}
    \item []
     \begin{minted}[linenos,autogobble]{fortran}
             DIMENSION A(11)
             READ A
      2      DO 3,8,11 J=1,11
      3      I=11-J
             Y=SQRT(ABS(A(I+1)))+5*A(I+1)**3
             IF (400>=Y) 8,4
      4      PRINT I,999.
             GOTO 2
      8      PRINT I,Y
      11     STOP
     \end{minted}
    \end{itemize}
\end{frame}


\begin{frame}[fragile]{Orígenes}
    \textbf{Algoritmo TPK en FORTRAN I}
    \begin{itemize}
    \item []  
     \begin{minted}[linenos,autogobble]{fortran}
      C      THE TPK ALGORITHM
      C      FORTRAN I STYLE
             FUNF(T)=SQRTF(ABSF(T))+5.0*T**3
             DIMENSION A(11)
        1    FORMAT(6F12.4)
             READ 1,A
             DO 10 J=1,11
             I=11-J
             Y=FUNF(A(I+1))
             IF(400.0-Y)4,8,8
        4    PRINT 5,I
        5    FORMAT(I10,10H TOO LARGE)
             GOTO 10
        8    PRINT 9,I,Y
        9    FORMAT(I10,F12.7)
       10    CONTINUE
             STOP 52525
     \end{minted}
    \end{itemize}
\end{frame}


% La estandarización ---------------------------------------------------------80
\subsection{La estandarización}
\begin{frame}[fragile]{La estándarización}
  \begin{itemize}[<+(1)->]
    \item[1962] El primer comité de estandarización de la ASA (Ahora ANSI) se re reune.
    \item[1966] Publicación del ANSI X3.91966 (FORTRAN 66), el primer estándar.
    \item[1978] Publicación del ANSI X3.91978 (FORTRAN 77), tambien publicado como ISO 1539:1980.
    \item[1991] ISO/IEC 1539:1991 (Fortran 90)
    \item[1997] ISO/IEC 1539­1:1997 (Fortran 95)
    \item[2004] ISO/IEC 1539­1:2004 (Fortran 2003)
    \item[2010] ISO/IEC 1539­1:2010 (Fortran 2008)
  \end{itemize}
\end{frame}


\begin{frame}[fragile]{La estándarización}
    \textbf{Algoritmo TPK en FORTRAN 77}
    \begin{itemize}
    \item []   
     \begin{minted}[linenos,autogobble]{fortran}
             PROGRAM TPK
     C       THE TPK ALGORITHM
     C       FORTRAN 77 STYLE
             REAL A(0:10)
             READ (5,*) A
             DO 10 I = 10, 0, -1
                     Y = FUN(A(I))
                     IF ( Y . LT. 400) THEN
                              WRITE(6,9) I,Y
       9                      FORMAT(I10. F12.6)
                     ELSE
                              WRITE (6,5) I
       5                      FORMAT(I10,' TOO LARGE')
                     ENDIF
      10    CONTINUE
            END

            REAL FUNCTION FUN(T)
            REAL T
            FUN = SQRT(ABS(T)) + 5.0*T**3
            END
     \end{minted}
    \end{itemize}
\end{frame}


\begin{frame}[fragile]{La estándarización}
    \textbf{Algoritmo TPK en FORTRAN 90}
    \begin{itemize}
    \item []  
     \begin{minted}[linenos,autogobble]{fortran}
            PROGRAM TPK
     !      The TPK Algorithm
     !      Fortran 90 style
            IMPLICIT NONE
            INTEGER          :: I
            REAL                       :: Y
            REAL, DIMENSION(0:10)      :: A
            READ (*,*) A
            DO I = 10, 0, -1           ! Backwards
                    Y = FUN(A(I))
                    IF ( Y < 400.0 ) THEN
                           WRITE(*,*) I, Y
                    ELSE
                           WRITE(*,*) I, ' Too large'
                    END IF
            END DO
            CONTAINS                   ! Local function
                    FUNCTION FUN(T)
                    REAL  :: FUN
                    REAL, INTENT(IN) :: T
                    FUN = SQRT(ABS(T)) + 5.0*T**3
                    END FUNCTION FUN
            END PROGRAM TPK
     \end{minted}
    \end{itemize}
\end{frame}

\begin{frame}[fragile]{La estándarización}
    \textbf{Algoritmo TPK en FORTRAN 90}
    \begin{itemize}
    \item []  
     \begin{minted}[linenos,autogobble]{fortran}
            module Functions  
            public :: fun
            contains
               function fun(t) result (r)
                  real, intent(in) :: t
                  real  :: r
                  r = sqrt(abs(t)) + 5.0*t**3
               end function fun
            end module Functions

            program TPK
      !     The TPK Algorithm
      !     F95 style
            use Functions
            integer       :: i
            real                  :: y
            real, dimension(0:10) :: a
            read *, a
            do i = 10, 0, -1      ! Backwards
               y = fun(a(i))
               if ( y < 400.0 ) then
                  print *, i, y
               else
                  print *, i, " Too large"
               end if
            end do
            end program TPKTPK
     \end{minted}
    \end{itemize}
\end{frame}


% Actualidad -----------------------------------------------------------------80
\subsection{En la actualidad}
\begin{frame}[fragile]{En la actualidad ...}
  \begin{quote}
    ``I don't know what the programming language of the year 2000 will look like, but I know it will be called FORTRAN.''\\
    Charles Anthony Richard Hoare\\
    Circa 1982
  \end{quote}
\end{frame}


\begin{frame}[fragile]{En la actualidad ...}
  \textbf{Aplicaciones}
  \begin{itemize}[<+(1)->]
    \item Predicción del clima
    \item Análisis de datos de sísmicos para la exploración de depositos de gas y petroleo
    \item Análisis fianciero
    \item Simulacion de choques vehiculares
    \item Análisis de datos de sondas espaciales
    \item Modelación de armas nucleares
    \item Dinámica de fluidos computacionales
    \item ``Numerical Wind Tunnel''
  \end{itemize}
\end{frame}


\begin{frame}[fragile]{En la actualidad ...}
  \begin{itemize}[<+(1)->]
    \item Intel
    \item IBM
    \item NVIDIA
  \end{itemize}
\end{frame}


% Futuro ---------------------------------------------------------------------80
\subsection{El futuro}
\begin{frame}[fragile]{El futuro}
  \begin{itemize}[<+(1)->]
    \item \textbf{WG5: }\url{https://wg5-fortran.org/}
    \item \textbf{Fortran Wiki: }\url{http://fortranwiki.org/}
    \item \textbf{WG5: }\url{https://wg5-fortran.org/}
  \end{itemize}
\end{frame}

