%-----------------------------------------------------------------------------80
% SECTION TITLE
%-----------------------------------------------------------------------------80

\section{Procedimientos}  

%-----------------------------------------------------------------------------80
% CONTENT
%-----------------------------------------------------------------------------80

%Definición-------------------------------------------------------------------80

\subsection{Introducción}

\begin{frame}[fragile]{Procedimientos}
 \begin{itemize}[<+(0)->]
 \item Qué es una unidad programática?
 \item Existen cuatro tipos de unidades programáticas:
 \item [-]\emph{program}, unidad programática principal.
 \item [-]\emph{subroutine}, contempla instrucciones ejecutables.
 \item [-]\emph{function}, tipo particular de subrutina.
 \item [-]\emph{module}, contempla instrucciones de declaración de variables, su inicialización e interfaces entre funciones y subrutinas.
 \item El programa principal, las subrutinas y funciones son llamados procedimientos, siendo estos de orden jerárquico.
 \item Un procedimiento es interno si está definido dentro de una unidad programática, que puede ser "hospedante" o bien un módulo; mientras que los externos no están contenidas en unidad programática alguna.
 \end{itemize}
\end{frame} 