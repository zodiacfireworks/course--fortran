-----------------------------------------------------------------------------80
% SECTION TITLE
%-----------------------------------------------------------------------------80

\section{Funciones}  

%-----------------------------------------------------------------------------80
% CONTENT
%-----------------------------------------------------------------------------80


\begin{frame}[fragile]{Funciones}
 \begin{itemize}[<+(0)->]
  \item Una función es un tipo particular de subrutina, a la que se accede directiamente sin utilizar la instrucción CALL.
  \item Algunas ventajas del uso de funciones es
  \item [-]Permite el uso de una acción, incluidas las funciones intrínsecas.
  \item [-]Permite dedicar una parte del código exclusivamente a resolver un problema.
  \item [-]Permite construir una biblioteca de funciones o módulos para resolver sub-problemas particulares para un mejor orden y mayor efectividad. 
  \item La estructura sintáxica de una función es la siguiente:
        \begin{minted}[linenos,autogobble]{fortran}
        [<tipo>] FUNCTION <nombre> [(<argumentos ficticios>)] [result (<resultado>)]
        : ! declaración función (nombre o resultado)
        : ! declaración argumentos ficticios
        : ! declaración objetos locales
        : ! instrucciones ejecutables
        END FUNCTION <nombre>
        \end{minted}
 \end{itemize}
\end{frame}