%-----------------------------------------------------------------------------80
% SECTION TITLE
%-----------------------------------------------------------------------------80

\section{Funciones}  

%-----------------------------------------------------------------------------80
% CONTENT
%-----------------------------------------------------------------------------80


\begin{frame}[fragile]{Funciones}
 \begin{itemize}[<+(0)->]
  \item Una función es un tipo particular de subrutina, a la que se accede directiamente sin utilizar la instrucción CALL.
  \item Algunas ventajas del uso de funciones son
  \item [-]Permite el uso de una acción, incluidas las funciones intrínsecas.
  \item [-]Permite dedicar una parte del código exclusivamente a resolver un problema.
  \item [-]Permite construir una biblioteca de funciones o módulos para resolver sub-problemas particulares para un mejor orden y mayor efectividad. 
  \item La estructura sintáxica de una función es la siguiente:
        \begin{minted}[linenos,autogobble]{fortran}
        [<tipo>] FUNCTION <nombre> [(<argumentos ficticios>)] [result (<resultado>)]
        : ! declaración función (nombre o resultado)
        : ! declaración argumentos ficticios
        : ! declaración objetos locales
        : ! instrucciones ejecutables
        END FUNCTION <nombre>
        \end{minted}
 \end{itemize}
\end{frame}

\begin{frame}[fragile]{Funciones} 
    \begin{table}[]
    \centering
    \label{Tabla_funcionesintr1}
    \resizebox{9cm}{!} {
    \begin{tabular}{|c|c|c|}
    \hline
    Función              & Descripción                                  & Ejemplo         \\ \hline
    abs                  & |x|, |z|, |n|                                & Y=ABS(X)        \\ \hline
    sqrt                 & ($\sqrt{x}$, x $\geq$ 0),                    & Y=SQRT(X)       \\ \hline
    int                  & Parte entera de un real x                    & J=INT(X)        \\ \hline  
    cos                  & $(\cos x, x \in \mathbb{R})$                 & Y=COS(X)        \\ \hline 
    sin                  & $(\sin x, x \in \mathbb{R})$                 & Y=SIN(X)        \\ \hline 
    tan                  & $(\tan x, x \in \mathbb{R})$                 & Y=TAN(X)        \\ \hline 
    acos                 & $(\arccos x, x \in \mathbb{R})$              & Y=ACOS(X)       \\ \hline 
    asin                 & $(\arcsin x, x \in \mathbb{R})$              & Y=ASIN(X)       \\ \hline 
    atan                 & $(\arctan x, x \in \mathbb{R})$              & Y=ATAN(X)       \\ \hline
    atan2                & $(\arctan (A/B), A, B \in \mathbb{R})$       & Y=ATAN2(A,B)    \\ \hline 
    exp                  & $(\exp x, x \in \mathbb{R})$                 & Y=EXP(X)        \\ \hline 
    log                  & $(\log x, x > 0)$                            & Y=LOG(X)        \\ \hline 
    log10                & $(\log_{10} x, x > 0$                        & Y=LOG10(X)      \\ \hline 
    \end{tabular}}
    \caption*{Funciones intrínsecas}
    \end{table}
\end{frame}

\begin{frame}[fragile]{Funciones}
\textbf{Funciones genéricas}
 \begin{itemize}[<+(1)->]
  \item Son aquellas funciones intrínsecas que pueden ser llamadas con cualquier argumento numérico de cualquier tipo KIND.
        \begin{minted}[linenos,autogobble]{fortran}
            PROGRAM Functions01
            IMPLICIT NONE
            COMPLEX :: C=(1,1)
            REAL    :: R=10.9
            INTEGER :: I=-27
              PRINT *,ABS(C)
              PRINT *,ABS(R)
              PRINT *,ABS(I)
            END PROGRAM Functions01
        \end{minted}
 \end{itemize}
\end{frame}

\begin{frame}[fragile]{Funciones}
 \begin{itemize}[<+(0)->]
  \item [] \textbf{Funciones elementales}
  \item Estas funciones trabajan con argumentos escalares y arreglos, es decir, ya sean simple o multiple valuados.
  \vspace{0.15cm}
  \item[]
      \begin{minted}[linenos,autogobble]{fortran}
      PROGRAM Functions02
      REAL , DIMENSION(5) :: X = (/1.0,2.0,3.0,4.0,5.0/)
        PRINT *,' Sine of ', X ,' = ',SIN(X)
      END PROGRAM Functions02
      \end{minted}
  \item [] \textbf{Funciones de transformación}
  \item Son aquellas cuyos argumentos son arreglos, actuando sobre los mismos.
  \vspace{0.15cm}
  \item [] 
        \begin{minted}[linenos,autogobble]{fortran}
            PROGRAM Functions0304     
            IMPLICIT NONE
            REAL , DIMENSION(5) :: X = (/1.0,2.0,3.0,4.0,5.0/)
            ! Transformational function dotproduct
                PRINT *,' Dot product of X with X is'
                PRINT *,' ',DOT_PRODUCT(X,X)
            ! Transformational function sum
                PRINT *,' Sum of ', X ,' = ',SUM(X)
            END PROGRAM Functions0304
        \end{minted}
 \end{itemize}
\end{frame}

\begin{frame}[fragile]{Funciones predefinidas}
 \textbf{Funciones de consulta}
 \begin{itemize}[<+(1)->]
  \item Estas funciones retornan información sobre sus argumentos. Pueden subclasificarse en BIT, CHARACTER, NUMERIC, ARRAY, POINTER y ARGUMENT PRESENCE.
   \begin{table}[]
    \centering
    \label{Tabla_funciones_predefinidas}
    \resizebox{9cm}{!} {
      \begin{tabular}{|l|l|}
        \hline
        \textbf{Bit}                  & BIT\_SIZE                                 \\ \hline
        \textbf{Character}            & LEN                                       \\ \hline
        \textbf{Numeric}              & DIGITS, EPSILON, EXPONENT, FRACTION,      \\
                                      & HUGE, KIND, MAXEXPONENT, MINEXPONENT,     \\
                                      & NEAREST, PRECISION, RADIX, RANGE,         \\
                                      & RRSPACING, SCALE, SET\_EXPONENT,          \\
                                      & SELECTED\_INT\_KIND, SELECTED\_REAL\_KIND,\\ 
                                      & SPACING, TINY                             \\ \hline
        \textbf{Array}                & ALLOCATED, LBOUND, SHAPE, SIZE, UBOUND    \\ \hline  
        \textbf{Pointer}              & ASSOCIATED, NULL                          \\ \hline 
        \textbf{Argument}             & PRESENT                                   \\
        \textbf{present}              &                                           \\ \hline
      \end{tabular}}
    \end{table}
 \end{itemize}
\end{frame}

\begin{frame}[fragile]{Funciones predefinidas}
 \textbf{Transferencia y conversión de funciones}
 \begin{itemize}[<+(1)->]
  \item Estas funciones convierten datos de un tipo y KIND a otro tipo y otro KIND.
   \begin{table}[]
    \centering
    \label{Tabla_funciones_predefinidas}
    \resizebox{9cm}{!} {
      \begin{tabular}{|c|c|}
        \hline
        \textbf{Transferencia}  & ACHAR, AIMAG, AINT, ANINT, CHAR, CMPLX, CONJG,\\        
        \textbf{y}              & DBLE, IACHAR, IBITS, ICHAR, INT, LOGICAL,\\                                  \\
        \textbf{conversión}     & NINT, REAL, TRANSFER.                                                 \\ \hline
      \end{tabular}}
    \end{table}
 \end{itemize}
\end{frame}

\begin{frame}[fragile]{Funciones predefinidas}
 \textbf{Transferencia y conversión de funciones}
 \begin{itemize}[<+(1)->]
  \item Estas funciones convierten datos de un tipo y KIND a otro tipo y otro KIND.
   \begin{table}[]
    \centering
    \label{Tabla_funciones_predefinidas}
    \resizebox{9cm}{!} {
      \begin{tabular}{|c|c|}
        \hline
        \textbf{Transferencia}  & ACHAR, AIMAG, AINT, ANINT, CHAR, CMPLX, CONJG,\\        
        \textbf{y}              & DBLE, IACHAR, IBITS, ICHAR, INT, LOGICAL,\\                                  \\
        \textbf{conversión}     & NINT, REAL, TRANSFER.                                                 \\ \hline
      \end{tabular}}
    \end{table}
 \end{itemize}
\end{frame}