%%----------------------------------------------------------------------------80
%% Preamble
%%----------------------------------------------------------------------------80
\documentclass[12pt, twoside, a4paper, final]{article}
\usepackage[some]{background}
\usepackage[utf8]{inputenc}
\usepackage[spanish]{babel}
\usepackage[T1]{fontenc}
\usepackage{pgfcalendar}
\usepackage{pdflscape}
\usepackage{colortbl}
\usepackage{enumitem}
\usepackage{fancyhdr}
\usepackage{graphicx}
\usepackage{geometry}
\usepackage{hyperref}
\usepackage{multirow}
\usepackage{pgfplots}
\usepackage{tabularx}
\usepackage{titlesec}
\usepackage{amsmath}
\usepackage{caption}
\usepackage{charter}
\usepackage{hologo}
\usepackage{lipsum}
\usepackage{xcolor}
\usepackage{array}
\usepackage{avant}


%%----------------------------------------------------------------------------80
%% Settings
%%----------------------------------------------------------------------------80
% Hyperref package settings --------------------------------------------------80
\hypersetup{
  pdftitle={Programación con Fortran - Nivel Intermedio},
  pdfauthor={Mart\'{i}n Josemar\'{i}a Vuelta Rojas},
  pdfpagelayout=OneColumn,
  pdfnewwindow=true,
  pdfdisplaydoctitle=true,
  pdfstartview=XYZ,
  plainpages=false,
  hidelinks=false,
  unicode=true,
  bookmarksnumbered=true,
  bookmarksopen=true,
  bookmarksopenlevel=3,
  breaklinks=true,
  colorlinks=true,
  pdfborder={0 0 0}
}


% caption poackage settings --------------------------------------------------80
% \captionsetup[figure]{
%   labelfont=bf,
%   justification=centering
% }


% pgfplots package settings --------------------------------------------------80
\pgfplotsset{compat=1.15}
% \usepgfplotslibrary{dateplot}
% \usepgfplotslibrary{groupplots}


%%----------------------------------------------------------------------------80
%% Customizations
%%----------------------------------------------------------------------------80
% Cover shapes ---------------------------------------------------------------80
\backgroundsetup{
  scale=1,
  angle=0,
  opacity=1,
  contents={
  \begin{tikzpicture}[
      remember picture,
      overlay
    ]
    \path[
      fill=red-softbutterfly
    ] (current page.north west) rectangle (-8.5, 0);
    \path[
      fill=blue-softbutterfly
    ] (current page.south west) rectangle (-8.5, 0);
  \end{tikzpicture}
  }
}


% Fonts customization --------------------------------------------------------80
% Customizing chapter heading
% Font family: sans serif
\titleformat{\title}[display]{
  \normalfont\sffamily\Huge\bfseries\color{grey-800}
}{}{0pt}{}

% Customizing section heading
% Font family: sans serif
\titleformat{\section}{
  \normalfont\sffamily\large\bfseries\color{grey-800}
}{\thesection}{1em}{}

% Customizing section heading
% Font family: sans serif
\titleformat{\subsection}{
  \normalfont\sffamily\normalsize\bfseries\color{grey-800}
}{\thesubsection}{1em}{}


% Page geometry customization ------------------------------------------------80
% Customizing header spacings
\geometry{
  a4paper,
  headheight=25pt,
  headsep=12pt,
}


%%----------------------------------------------------------------------------80
%% Command definitions
%%----------------------------------------------------------------------------80
\makeatletter

% Colors ---------------------------------------------------------------------80
% Softbutterfly colors
\definecolor{red-softbutterfly}{HTML}{C83737}
\definecolor{blue-softbutterfly}{HTML}{162D50}

% Material design colors
\definecolor{white}{HTML}{FFFFFF}

\definecolor{red-500}{HTML}{F44336}
\definecolor{red-A700}{HTML}{D50000}

\definecolor{blue-50}{HTML}{E3F2FD}
\definecolor{blue-100}{HTML}{BBDEFB}
\definecolor{blue-800}{HTML}{1565C0}

\definecolor{grey-50}{HTML}{FAFAFA}
\definecolor{grey-800}{HTML}{424242}
\definecolor{grey-900}{HTML}{212121}

% Command for setting global color
\newcommand{\globalcolor}[1]{
  \color{#1}\global\let\default@color\current@color
}


% Fonts definitions ----------------------------------------------------------80
% Redefining default document font family
% Font family: serif
\renewcommand{\familydefault}{\rmdefault}


% Parragraph spacings --------------------------------------------------------80
% \setlength{\parindent}{4em}
% \setlength{\parskip}{0.5em}


% Table columns types --------------------------------------------------------80
\newcolumntype{L}[1]{>{\raggedright\let\newline\\\arraybackslash\hspace{0pt}}m{#1}}
\newcolumntype{C}[1]{>{\centering\let\newline\\\arraybackslash\hspace{0pt}}m{#1}}
\newcolumntype{R}[1]{>{\raggedleft\let\newline\\\arraybackslash\hspace{0pt}}m{#1}}

% Color box decoration -------------------------------------------------------80
\newcommand{\headerbox}[2][]{
  \tikz[overlay]
  \node[fill=blue!20,inner sep=4pt, anchor=text, rectangle, rounded corners=1mm, #1] {#2};
  \phantom{#2}
}

\newcommand{\scorebox}[2][]{
  \tikz[overlay]
  \node[fill=blue!20,inner sep=10pt, anchor=text, rectangle, rounded corners=1mm, #1] {#2};
  \phantom{#2}
}

% Clear double page definition -----------------------------------------------80
% Use of "This page is intentionally left in blank." for better UX
% Use of "Página intencionalmente dejada en blanco." for better UX
\def\cleardoubleintentionalpage{
  \clearpage%
  \if@twoside%
  \ifodd%
    \c@page%
  \else%
    \vspace*{\fill}%
    \begin{center}%
    % For spanish:
    \rm\emph{Página intencionalmente dejada en blanco.}
    % For english:
    % \rm\emph{This page is intentionally left in blank.}
    \end{center}%
    \vspace{\fill}%
    \thispagestyle{empty}%
    \newpage%
    \if@twocolumn%
    \hbox{}%
    \newpage%
    \fi%
  \fi%
  \fi%
}

% Completely white page
\def\cleardoublepage{
  \clearpage%
  \if@twoside%
  \ifodd%
    \c@page%
  \else%
    \vspace{\fill}%
    \thispagestyle{empty}%
    \newpage%
    \if@twocolumn%
    \hbox{}%
    \newpage%
    \fi%
  \fi%
  \fi%
}

% Flip margin ----------------------------------------------------------------80
\newcommand*{\flipmargins}{%
  \clearpage
  \setlength{\@tempdima}{\oddsidemargin}%
  \setlength{\oddsidemargin}{\evensidemargin}%
  \setlength{\evensidemargin}{\@tempdima}%
  \if@reversemargin
  \normalmarginpar
  \else
  \reversemarginpar
  \fi
}


% Page style definition ------------------------------------------------------80
% Plain Style
\fancypagestyle{plain}{%
  \fancyhead{}%
  \renewcommand{\headrulewidth}{0pt}
}

% Fancy Style
\fancypagestyle{fancy}{
  \fancyhf{}
  \renewcommand{\headrulewidth}{1pt}%
  \renewcommand{\footrulewidth}{0pt}%
  \fancyhead[EL]{
    % Make use of your own phrase
    Fortran
  }
  \fancyhead[OR]{
    % Make use of your own phrase
    \noindent\softbutterflylogo{3.5cm}\par
  }
  \fancyfoot[C]{
  {
  \begin{centering}
    \thepage
  \end{centering}
  }
  }
}

% noindent environment -------------------------------------------------------80
\newenvironment{noindented}{
  \par\setlength{\parindent}{0em}
}
{\par}

% Some useful symbols --------------------------------------------------------80
% Registered Mark
\newcommand{\registeredmark}{\textsuperscript{\small{\textregistered}}}

% SoftButterfly logo
\newcommand{\softbutterflylogo}[1]{
  \includegraphics[width=#1]{../resources/SoftButterfly-LaTeX-Logo.pdf}
}


\makeatother


%%----------------------------------------------------------------------------80
%% Document
%%----------------------------------------------------------------------------80
% Global document settings ---------------------------------------------------80
\AtBeginDocument{
  \globalcolor{grey-900}
}


% Document body --------------------------------------------------------------80
\begin{document}
  % BEGIN: Cover pager
  \begin{titlepage}
    \BgThispage
    \newgeometry{left=5cm,top=6cm,bottom=6cm,right=3.5cm}
    \begin{noindented}
      \setlength{\parskip}{0.5em}
      {
        \softbutterflylogo{7cm}
        \vspace{-1em}
      }
      \par
      {%
        \color{red-softbutterfly}
        \makebox[0pt][l]{
          \rule{1.3\textwidth}{1pt}
        }%
      }
      \par
      {
        \color{blue-softbutterfly}
        \textsf{
          \textbf{
            \textsc{Propuesta técnica de cruso de capacitación}
          }
        }
      }
      \par
      \vfill
      {
        \huge
        \textsf{Programación en Fortran}
      }
      \par
      {
        \large
        \textsf{Nivel intermedio}
      }
      \par
      \vfill
      {
        \color{blue-softbutterfly}
        \textsf{Sílabo elaborado por}
      }
      \par
      {
        \large
        \textsf{Martín Josemaría Vuelta Rojas}
      }
      \vskip\baselineskip
      \textsf{Lima, \today}
    \end{noindented}
    \afterpage{
      \cleardoubleintentionalpage
    }
  \end{titlepage}
  \restoregeometry

  % BEGIN: Report content
  \begin{noindented}
  {
    \huge\bfseries
    \textsf{%
      \textsc{Programación en Fortran}
    }
  }
  \par
  {
    \large
    \textsf{%
      \textsc{Nivel intermedio}
    }
  }
  \vspace{1em}
\end{noindented}

  %%----------------------------------------------------------------------------80
%% Section title
%%----------------------------------------------------------------------------80
\section{Introducción}


%%----------------------------------------------------------------------------80
%% Section Content
%%----------------------------------------------------------------------------80
Fortran es un lenguaje de programación desarrollado en los años 50 y activamente utilizado desde entonces en aplicaciones científicas y análisis numérico. Ha sido ampliamente adoptado por la comunidad científica para escribir aplicaciones con cómputos intensivos de alto rendimiento.

Desde 1958 ha pasado por varias versiones, entre las que destacan FORTRAN II, FORTRAN IV, FORTRAN 77, Fortran 90, Fortran 95, Fortran 2003 y Fortran 2008. Si bien el lenguaje era inicialmente un lenguaje imperativo, las últimas versiones incluyen elementos de la programación orientada a objetos.

El curso de Fortran (nivel básico), está orientado a que se aprendan los fundamentos de la programación en Fortran con énfasis en Fortran 90/95, por ser los estándares de mayor uso, y brindar un acercamiento a las nuevas técnicas de programación introducidas por los estándares más sin que ello afecte al rendimiento del las aplicaciones escritas con versiones anteriores.

  %%----------------------------------------------------------------------------80
%% Section title
%%----------------------------------------------------------------------------80
\section{Detalles}


%%----------------------------------------------------------------------------80
%% Section Content
%%----------------------------------------------------------------------------80
\begin{description}
  \item[Docente]\hfill\\
    Martín Josemaría Vuelta Rojas

    \begin{itemize}
      \item Linkedin: \url{www.linkedin.com/in/martinvuelta}
      \item GitHub: \url{www.github.com/zodiacfireworks}
      \item Correo electrónico: \href{mailto:martin.vuelta@gmail.com}{\texttt{martin.vuelta@gmail.com}}
    \end{itemize}

  \item[Nivel]\hfill\\
    Básico.

  \item[Metodología]\hfill
    \begin{itemize}
      \item Exposiciones dialogadas utilizando, apuntes, elementos de proyección fija.
      \item Talleres prácticos grupales en la elaboración de programas.
      \item Las clases serán eminentemente prácticas en una relación 70\% práctica - 30\% teórica.
    \end{itemize}

  \item[Duración]\hfill\\
    16 horas\footnote{Se sugiere una división en 4 sesiones de 4 horas con un receso de 10 minutos en cada sesión.}

  \item[Materiales a entregar]\hfill
    \begin{itemize}
      \item Manual de cada clase (PDF)
      \item Diapositivas de cada clase (PDF)
      \item \texttt{Cheatsheet} de Fortran (Impreso)
    \end{itemize}

  \item[Requerimientos]\hfill
    \begin{itemize}
      \item Proyector (con cable HDMI)
      \item Acceso a una conexión de Internet (WiFi o Cableado)
      \item Cada alumno deberá contar con su propio equipo para trabajar
    \end{itemize}
\end{description}

  %%----------------------------------------------------------------------------80
%% Section title
%%----------------------------------------------------------------------------80
\section{Temario}


%%----------------------------------------------------------------------------80
%% Section Content
%%----------------------------------------------------------------------------80
\begin{description}
  \item[Sesión 1. Programación Orientada a Objetos: Re-usabilidad]\hfill
    \begin{enumerate}
      \item Introducción
      \item Objetos en fortran 90/95
      \item Objetos en fortran 2003
      \item Polimorfismo en Fortran 2003
      \item Polimorfismo de procedimientos
      \item Polimorfismo de procedimientos asociados a un tipo
      \item Herencia
      \item Sobre escritura de Procedimientos
      \item Funciones como procedimientos asociados a un tipo
      \item Ocultación de información
      \item Sobrecarga de tipos
      \item Proyecto 1: Simulaciones de partículas
    \end{enumerate}

  \item[Sesión 2. Programación Orientada a Objetos: Polimorfismo de datos]\hfill
    \begin{enumerate}
      \item Polimorfismo de datos
      \item Objetos polimorfos
      \item Caso de estudio: Polimorfismo de datos en listas enlazadas
      \item Tipos abstractos y enlaces diferidos
      \item Proyecto 2: Listas enlazadas
    \end{enumerate}

  \item[Sesión 3. Introducción a la interoperabilidad]\hfill
    \begin{enumerate}
      \item Interoperabilidad
      \item Interoperabilidad en Fortran 90/95
      \item Interoperabilidad en Fortran 2003
      \item Interoperabilidad con C
      \item Interoperabilidad con C++
      \item Interoperabilidad con Python
      \item Proyecto 3: Interfaces gráficas
    \end{enumerate}

  \item[Sesión 4. Introducción a la computación de Alto Rendimiento]\hfill
    \begin{enumerate}
      \item Computación de Alto Rendimiento
      \item Nociones básicas de optimización
      \item Técnicas básicas de perfilaje
      \item Open MP
      \item MPI
      \item Método híbrido: Open MP y Open MPI
      \item Proyecto 4: Modelo de Ising
    \end{enumerate}
\end{description}

\end{document}
