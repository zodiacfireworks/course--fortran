%%----------------------------------------------------------------------------80
%% Section title
%%----------------------------------------------------------------------------80
\section{Temario}


%%----------------------------------------------------------------------------80
%% Section Content
%%----------------------------------------------------------------------------80
\begin{description}
  \item[Sesión 1. Introducción]\hfill
    \begin{enumerate}
      \item Fortran: Historia y estado actual
      \item Preparación del entorno de trabajo
      \item Conceptos básicos: \texttt{Hello world!} y el proceso de compilación
        \begin{itemize}
          \item \texttt{Hello world!}
          \item Compilación
          \item Enlazado
          \item Productos finales
        \end{itemize}
      \item Estructura de un programa en Fortran
      \item Formatos de escritura en Fortran
        \begin{itemize}
          \item Formato fijo
          \item Formato libre
        \end{itemize}
      \item Expresividad: El algoritmo TPK
    \end{enumerate}

  \item[Sesión 2. Instrucciones básicas]\hfill
    \begin{enumerate}
      \item Variables y tipos de datos
        \begin{itemize}
          \item Variables y Constantes
          \item Cadenas de caracteres
          \item Valores lógicos
          \item Enteros
          \item Reales (Precisión simple y doble)
          \item Complejo
          \item La clausula \texttt{IMPLICIT}
          \item Tipos derivados
        \end{itemize}

      \item Operaciones elementales
        \begin{itemize}
          \item Aritmética
          \item Comparación
          \item Lógica
          \item Funciones intrínsecas (\textit{builtin functions})
          \item Operaciones con caracteres
        \end{itemize}

      \item Estructuras de control
        \begin{itemize}
          \item Condiciones: \texttt{IF}
          \item Repeticiones
            \begin{itemize}
              \item \texttt{DO}
              \item \texttt{DO WHILE}
              \item \texttt{DO FROM X TO Y}
            \end{itemize}
          \item Selección: \texttt{SWITCH CASE}
          \item \textit{The infamous} \texttt{GO TO}
        \end{itemize}

      \item Instrucciones básicas de lectura y escritura de datos
        \begin{itemize}
          \item Escritura sobre la pantalla
          \item Entrada de datos por el usuario
          \item Entrada de datos por linea de comandos (Fortran 2003)
        \end{itemize}

      \item Proyecto 1: Proyecto Euler
    \end{enumerate}

  \item[Sesión 3. Arreglos / Lectura y escritura de archivos] \hfill
    \begin{enumerate}
      \item Declaración de arreglos
        \begin{itemize}
          \item Asignación de valores a arreglos
          \item Declaración dinámica de arreglos
        \end{itemize}
      \item Asignación en arreglos
        \begin{itemize}
          \item Subarreglos
          \item Expresiones de asignación y operaciones aritméticas
        \end{itemize}
      \item Instrucciones y operaciones exclusivas de arreglos
        \begin{itemize}
          \item Instrucciones de control
          \item Funciones intrínsecas
        \end{itemize}
      \item Operaciones matriciales

      \item Modificadores de formato
      \item Lectura y escritura de archivos
        \begin{itemize}
          \item Apertura de archivos
          \item Lectura y escritura en archivos
        \end{itemize}
      \item Ejecución de comandos del sistema operativo (Fortran 2008)

      \item Proyecto 2: Lectura de archivos de configuración
    \end{enumerate}

  \item[Sesión 4. Procedimientos / Módulos] \hfill
    \begin{enumerate}
      \item Subrutinas
        \begin{itemize}
          \item Argumentos ficticios
          \item Objetos locales
          \item Subrutinas internas
        \end{itemize}

      \item Funciones

      \item Utilidades
        \begin{itemize}
          \item Argumentos por nombre
          \item Argumentos opcionales
          \item Recursividad
          \item \texttt{PURE} \textit{keyword}
        \end{itemize}

      \item Módulos
        \begin{itemize}
          \item Datos y Objetos Compartidos
          \item Procedimientos de módulo
        \end{itemize}

      \item Interfaces genéricas
        \begin{itemize}
          \item Interfaces genéricas con procedimientos
          \item Interfaz operador
          \item Interfaz de asignación
        \end{itemize}

      \item Proyecto 3: Manipulación de fechas y series de tiempo
    \end{enumerate}

  \item[Sesión 5: Tópicos adicionales] \hfill
    \begin{enumerate}
      \item Punteros
      \item Programación orientada a objetos
      \item Tópicos adicionales
      \begin{itemize}
        \item Opciones de compilador
        \item \textit{Makefiles}
        \item Control de versiones
      \end{itemize}
      \item Proyecto 4: Derivación e integración numérica
    \end{enumerate}
\end{description}
