%%----------------------------------------------------------------------------80
%% Section title
%%----------------------------------------------------------------------------80
\section{Introducción}


%%----------------------------------------------------------------------------80
%% Section Content
%%----------------------------------------------------------------------------80
Fortran es un lenguaje de programación desarrollado en los años 50 y activamente utilizado desde entonces en aplicaciones científicas y análisis numérico. Ha sido ampliamente adoptado por la comunidad científica para escribir aplicaciones con cómputos intensivos de alto rendimiento.

Desde 1958 ha pasado por varias versiones, entre las que destacan FORTRAN II, FORTRAN IV, FORTRAN 77, Fortran 90, Fortran 95, Fortran 2003 y Fortran 2008. Si bien el lenguaje era inicialmente un lenguaje imperativo, las últimas versiones incluyen elementos de la programación orientada a objetos.

El curso de Fortran (nivel básico), está orientado a que se aprendan los fundamentos de la programación en Fortran con énfasis en Fortran 90/95, por ser los estándares de mayor uso, y brindar un acercamiento a las nuevas técnicas de programación introducidas por los estándares más sin que ello afecte al rendimiento del las aplicaciones escritas con versiones anteriores.
