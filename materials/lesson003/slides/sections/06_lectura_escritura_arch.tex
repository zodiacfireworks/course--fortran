%-----------------------------------------------------------------------------80
% SECTION TITLE|
%-----------------------------------------------------------------------------80

\section{Lectura y escritura de archivos}  

%-----------------------------------------------------------------------------80
% CONTENT
%-----------------------------------------------------------------------------80

\subsection{Apertura/Lectura y escritura de archivos}

\begin{frame}[fragile]{Apertura de archivos} 
\textbf{Archivos}    
      \begin{itemize}[<+(1)->]
        \item Un archivo es un objeto que guarda o contiene datos, información o instrucciones en un espacio de memoria.
        \item Entre ellos destacan:
        \item [-] Archivos de texto. Son la fuente de lo programas, listas de compilación,etc. Pueden ser impresos y directamente leídos.
        \item [-] Archivos de datos. Existen en dos formas: los que pueden ser usados en un editor (archivo de texto) y aquellos que pueden ser usados un paquete o un programa (codificado).
        \item [-] Arcivos binarios. Secuencias de bits. No pueden ser impresos o examinados directamente, por ello necesitan de programas para su propósito.
      \end{itemize}
\onslide <7->\textbf{Archivo de datos en Fortran}
      \begin{itemize}[<+(1)->]
        \item Fortran permite asociar un número como unidad lógica con cualquier archivo para su creación, conexión y acceso.
      \end{itemize}  
\end{frame}

\begin{frame}[fragile]{Lectura y escritura de archivos}    
\textbf{Declaración OPEN}
  \begin{itemize}[<+(1)->]
    \item \emph{OPEN} permite crear o conectar una unidad numérica como especificador con un archivo.
    \item La sintaxis para la declaración es la siguiente:\\ 
      \centering OPEN (unidad de control, [lista de salida])
  \end{itemize}
\onslide <4->\textbf{Declaración CLOSE}
  \begin{itemize}[<+(2)->]
    \item \emph{CLOSE} permite borrar o desvincular la unidad numérica del archivo.
    \item La sintaxis para la declaración es la siguiente sintaxis:\\ 
      \centering CLOSE (unidad de control, [opciones])
  \end{itemize}
\onslide <7->\textbf{Declaración READ}
  \begin{itemize}[<+(3)->]
    \item \emph{READ} permite leer datos desde una referencia. Puede contener especificaciones del proceso de retorno de datos.
  \end{itemize}
\end{frame}

\begin{frame}[fragile]{Lectura y escritura de archivos}
 \textbf{Declaración WRITE}
  \begin{itemize}[<+(1)->]
    \item \emph{WRITE} permite mandar datos o escribirlos en unidad numerada que puede ser la pantalla, un archivo de disco u otro dispositivo externo.
    \item La opción \emph{PRINT} es una forma limitada de \emph{WRITE}.
    \item Tanto \emph{READ} como \emph{WRITE} tienen la siguiente sintaxis:\\ 
    \centering READ/WRITE (unidad de control, [lista de entrada/salida])
  \end{itemize}
 \onslide <5->\textbf{Declaración STATUS}
  \begin{itemize}[<+(2)->]
    \item \emph{STATUS} permite conocer el estado del archivo.
    \item Puede tomar cuatro valores: 
    \item [-] STATUS = 'OLD'
    \item [-] STATUS = 'NEW'
    \item [-] STATUS = 'SCRATCH'
    \item [-] STATUS = 'UNKNOWN'
  \end{itemize}
\end{frame}

\begin{frame}[fragile]{Lectura y escritura de archivos}
    \begin{minted}[linenos,autogobble]{fortran}
      PROGRAM File01
      IMPLICIT NONE
      REAL :: X
      CHARACTER (7) :: WHICH
          OPEN(UNIT=5,FILE='INPUT')
            DO
              WRITE(UNIT=6,FMT='('' DATA SET NAME, OR END'')')
              READ(UNIT=5,FMT='(A)') WHICH
              IF(WHICH == 'END') EXIT
              OPEN(UNIT=1,FILE=WHICH)
              READ(UNIT=1,FMT=100) X
              !...
          CLOSE(UNIT=1)
            END DO
      END PROGRAM File01
  \end{minted}
\end{frame}

\begin{frame}[fragile]{Lectura y escritura de archivos}
 \textbf{Opciones en declaraciones de entrada/salida}
  \begin{itemize}[<+(1)->]
   \item \textbf{UNIT:} El número del archivo que será abierto.
   \item \textbf{IOSTAT:} Especifica el estado I/O y designa una variable numérica para guardar un valor de acuerdo al estado del archivo. 
   \item [-] Un entero positivo, indicando una condición de error.
   \item [-] Cero, indicando que no existe un error, el término de un archivo o condición de registro durante la ejecución de \emph{READ}.
   \item \textbf{FILE:} Especifica el nombre del archivo.
   \item \textbf{STATUS:} Especifica el estado del archivo.
   \item \textbf{ACCESS:} Especifica la forma en la que se usará el archivo: secuencial (SEQUENTIAL) por defecto o aleatoria (DIRECT). 
  \end{itemize}
\end{frame}

\begin{frame}[fragile]{Lectura y escritura de archivos}
  \begin{itemize}[<+(0)->]
   \item \textbf{FORM:} 
   \item \textbf{RECL:} Variable entera o constante que especifica 
   \item \textbf{BLANK:} Especifica el reconocimiento de un espacio en blanco tomando los valores...
   \item [-] NULL, si son ignorados en la lectura.
   \item [-] ZERO, si los espacios son tratados como cero.
  \end{itemize}
\end{frame}

\begin{frame}[fragile]{Lectura y escritura de archivos} 
    \begin{itemize}[<+(0)->]
     \item [] A partir del siguiente programa:
      \vspace{0.1cm}
      \begin{minted}[linenos,autogobble]{fortran}
      PROGRAM matriz3x2
      INTEGER, DIMENSION(3,2)::A
      WRITE(*,*) "Introduzca los valores de la matriz A de talla 3x2"
      READ(*,*)A
      WRITE(*,*) "La matriz que usted ha introducido es:"
      WRITE(*,*) "Primera Fila",A(1,:)
      WRITE(*,*) "Segunda Fila",A(2,:)
      WRITE(*,*) "Tercera Fila",A(3,:)
      END PROGRAM matriz3x2
      \end{minted}

     \item [] podemos obtener, meditante la ejecución:
      \begin{mintedbashbox}
      [softbutterfly\@SB-PC] ~\$     
      Introduzca los valores de la matriz A de talla 3x2
      1,2,3,4 5 6
      La matriz que usted ha introducido es:
      Primera Fila           1           4
      Segunda Fila           2           5
      Tercera Fila           3           6
      [softbutterfly\@SB-PC]    
      \end{mintedbashbox}
    \end{itemize}
\end{frame}