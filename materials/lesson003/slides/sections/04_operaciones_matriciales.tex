%-----------------------------------------------------------------------------80
% SECTION TITLE|
%-----------------------------------------------------------------------------80

\section{Operadores matriciales}  

%-----------------------------------------------------------------------------80
% CONTENT
%-----------------------------------------------------------------------------80


\begin{frame}[fragile]{Operadores matriciales} 
    \begin{itemize}[<+(0->]
        \item Fortran permite asignar de valores en elementos específicos de un arreglo, bajo una determinada condición, con la instrucción WHERE. 
        \item La sintaxis es la siguiente \\
            \centering WHERE (<arreglo control>) <arreglo>=<expresión> 
            \leftline {donde los elementos de <arreglo control> y <arreglo> son tipo LOGICAL}
        \item [] Por ejemplo: \\ 
            Sea A un arreglo de $2 \times 2$ 
            $$
                A = \left( \begin{array}{cc}
                    100. & 10. \\
                    1. & 0. \end{array} \right)
            $$
        \item [] 
            \begin{minted}[linenos,autogobble]{fortran}
                :
                REAL, DIMENSION(2,2)::A,B
                :
                WHERE(A>0) B=log10(A)
                : 
            \end{minted}
        \item [] dará como resultado
            $$
                A = \left( \begin{array}{cc}
                    2. & 1. \\
                    0. & 0. \end{array} \right)
            $$   
    \end{itemize}
\end{frame}